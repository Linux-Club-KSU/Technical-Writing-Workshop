\documentclass{article}

\usepackage{graphicx}   % For images
\usepackage{xcolor}     % For colors
\usepackage{hyperref}   % For links
\usepackage{amsmath}    % For math
\usepackage{listings}   % For code blocks

\title{LaTeX Syntax Guide}
\author{}
\date{}

\begin{document}

\maketitle

This document demonstrates common LaTeX syntax.

\section{Headings}

\section{Level 1 Heading}
\subsection{Level 2 Heading}
\subsubsection{Level 3 Heading}

Headings are created using:
\begin{verbatim}
\section{...}
\subsection{...}
\subsubsection{...}
\end{verbatim}

\section{Bullet Lists}

Bullet points are created using the itemize environment:

\begin{verbatim}
\begin{itemize}
  \item First point
  \item Second point
\end{itemize}
\end{verbatim}

Example:

\begin{itemize}
  \item This is a bullet point
  \item This is another bullet point
    \begin{itemize}
      \item This is a sub point
        \begin{itemize}
          \item This is a sub sub point
        \end{itemize}
    \end{itemize}
\end{itemize}

\section{Numbered Lists}

Numbered lists use the enumerate environment:

\begin{verbatim}
\begin{enumerate}
  \item First
  \item Second
\end{enumerate}
\end{verbatim}

Example:

\begin{enumerate}
  \item Lists look like this
  \item They are created with enumerate
    \begin{enumerate}
      \item They can have sub lists
        \begin{enumerate}
          \item And even deeper levels
        \end{enumerate}
    \end{enumerate}
\end{enumerate}

\section{Text Formatting}

\textbf{Bold text}

\textit{Italic text}

\texttt{Monospace text}

\textcolor{blue}{Colored text}

You can also write inline code like this: \texttt{let x = 10}

\section{Links}

Links are created using hyperref:

\begin{verbatim}
\href{https://example.com}{This is a link}
\end{verbatim}

Example:

\href{https://example.com}{This is a link}

\section{Images}

Images require the graphicx package:

\begin{verbatim}
\includegraphics[width=0.3\textwidth]{image.png}
\end{verbatim}

Example:

\includegraphics[width=0.3\textwidth]{./Resume_Tex/profile.png}

\section{Code Blocks}

Code blocks can be created using the listings package:

\begin{verbatim}
\begin{lstlisting}[language=C]
#include <stdio.h>

int main() {
    printf("Hello World");
    return 0;
}
\end{lstlisting}
\end{verbatim}

Example:

\begin{lstlisting}[language=C]
#include <stdio.h>

int main() {
    printf("Hello World");
    return 0;
}
\end{lstlisting}

\section{Math}

Inline math: $a^2 + b^2 = c^2$

Displayed math:

\[
\int_0^1 x^2 \, dx
\]

\end{document}